\documentclass[11pt]{letter}
\usepackage[a4paper,left=2.5cm, right=2.5cm, top=1cm, bottom=1cm]{geometry}
\usepackage[osf]{mathpazo}
\usepackage{url}
\signature{Natalie Cooper on behalf of all coauthors}
\address{Natural History Museum, London \\ Cromwell Road \\ London, SW7 5BD \\ natalie.cooper@nhm.ac.uk}
\longindentation=0pt
\begin{document}

\begin{letter}{}
\opening{Dear Editors,}

Please find attached our letter, `Reconstructing the last known movements of one of Nature's giants', for consideration at Nature.

We get few opportunities to study the behaviour of rare and enigmatic species in the wild, especially wide-ranging marine species. 
Museum collections offer an opportunity to gain information on the movements and life-history of such species. 
In 1891 a 25m long female blue whale stranded in Wexford, Ireland. 
On July 13th 2017, this whale will take pride of place in the Natural History Museum's (NHM) Hintze Hall (we can provide a picture of this as a cover image if desired).
Ironically, we know very little about this animal, or the species as a whole. 

Here we use stable isotope analysis of the baleen of the NHM blue whale, along with a custom-built individual-based movement model, to identify the most likely movements of this individual over the last six years of her life. We identify two phases in the behaviour of the whale - short range seasonal migrations at northern latitudes, followed by isotopic signatures consistent with pregnancy, lactation in warmer waters, and then a return to her northern feeding grounds where she stranded. This provides crucial information about historical movements in blue whales at a point where the species was on the brink of extinction in the North Atlantic.

Blue whales were the first species that humans legislated to save, and represent a big conservation success story, although North Atlantic stocks have not recovered to the same extent as other whales in the region. There is also evidence that blue whales are shifting their ranges due to warming waters off Iceland. Thus understanding historical movements of North Atlantic blue whales has never been more vital if we wish to preserve them for future generations.
 
We look forward to hearing from you,


\closing{Yours sincerely,}


\end{letter}
\end{document}