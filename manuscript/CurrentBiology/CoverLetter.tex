\documentclass[11pt]{letter}
\usepackage[a4paper,left=2.5cm, right=2.5cm, top=1cm, bottom=1cm]{geometry}
\usepackage[osf]{mathpazo}
\usepackage{url}
\signature{Natalie Cooper on behalf of all coauthors}
\address{Natural History Museum, London \\ Cromwell Road \\ London, SW7 5BD \\ natalie.cooper@nhm.ac.uk}
\longindentation=0pt
\begin{document}

\begin{letter}{}
\opening{Dear Editors,}

Please find attached our letter, `Reconstructing the last known movements of one of Nature's giants', for consideration at Current Biology.

We get few opportunities to study the behaviour of rare and enigmatic species in the wild, especially wide-ranging marine species. 
The blue whale is an excellent example. 
Despite famously known as the largest animal ever to have lived, we know very little about the behavioral ecology species as a whole, particularly in the North Atlantic.

Museum collections offer an opportunity to gain information on the movements and life-history of such species. 
In 1891 a 25m long female blue whale stranded in Wexford, Ireland. 
From July 13th 2017, this whale has taken pride of place in the Natural History Museum London's (NHM) Hintze Hall (we can provide a picture of this as a cover image if desired).
Natural stable isotope markers have been widely used to infer information about individual-level behavior from animal tissues, but the level of behavioral detail that can be inferred has been limited by a lack of contextual data. 
Here we describe an entirely new method for inferring individual level movement data from stable isotope records from incremental tissues, based on recent advances in computer simulations. 
As a high profile case study, we apply our method to the baleen of the NHM blue whale ``Hope'' to identify the most likely movements of this individual over the last seven years of her life. 
We identify three phases in the behaviour of the whale - a juvenile period spent in warm waters, short range seasonal migrations at northern latitudes, followed by isotopic signatures consistent with pregnancy, calf rearing in warmer waters, and then a return to her northern feeding grounds where she stranded. 
Our approach can be applied to any marine animal with incrementally grown tissues, and unlocks a wealth of ecological information stored in animal tissues. 
In the context of blue whales, we provide the first multi-annual record of the behavior of juvenile blue whales in the North Atlantic, and reveal crucial information about historical movements in blue whales at a point where the species was on the brink of extirpation in the North Atlantic.

Blue whales were the first species that humans legislated to save, and represent a big conservation success story, although North Atlantic stocks have not recovered to the same extent as other whales in the region. 
There is also evidence that blue whales are shifting their ranges due to warming waters off Iceland. 
Thus understanding historical movements of North Atlantic blue whales has never been more vital if we wish to preserve them for future generations.
 
We look forward to hearing from you,


\closing{Yours sincerely,}


\end{letter}
\end{document}